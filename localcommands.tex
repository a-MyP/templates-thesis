%%%%%%%%%%%%%%%%%%%%%%%%%%%%%%%%%%%%%%%%%%%%%%%%%%%%
%%%           MyP-Commands  2019.10.05    PhD thesis + PDF-LaTeX       
%%%%%%%%%%%%%%%%%%%%%%%%%%%%%%%%%%%%%%%%%%%%%%%%%%%%   

%%%%%%%%%%%%%%%%%%%%%%%%%%%%%%%%
%% Extra complex long definitions (e.g. X-bar) are in:
%% 
%%% localextracommands/lec-...

%%%%%%%%%%%%%%%%%%%%%%%%%%%%%%%%
%% ABBREVIATIONS
%%%%%%%%%%%%%%%%%%%%%%%%%%%%%%%%

%%%%%%%%%%%%%%%%%%%%%%%%%%%%%%%%
% Abbreviations in German
% package needed: xspace
% Short space in German abbreviations: \,	
\newcommand{\dash}{\mbox{d.\,h.}\xspace}
\newcommand{\idR}{\mbox{i.\,d.\,R.}\xspace}
\newcommand{\ieS}{\mbox{i.\,e.\,S.}\xspace}
\newcommand{\iwS}{\mbox{i.\,w.\,S.}\xspace}
\newcommand{\su}{\mbox{s.\,u.}\xspace}
\newcommand{\ua}{\mbox{u.\,a.}\xspace}
\newcommand{\va}{\mbox{v.\,a.}\xspace}
\newcommand{\vgl}{\mbox{vgl.}\xspace}
\newcommand{\zB}{\mbox{z.\,B.}\xspace}
%\newcommand{\s}{s.~}
%not possibel: \dh --> d.\,h.

%%%%%%%%%%%%%%%%%%%%%%%%%%%%%%%%
%Abbreviations in English
\newcommand{\ao}{\mbox{a.o.}\xspace}	% among others
\newcommand{\cf}[1]{(cf.~#1)}	% confer = compare
\newcommand{\cfe}[1]{(cf.~(\ref{#1}))}	% compare + example
\newcommand{\ia}{i.a.}	% inter alia = among others
\newcommand{\ie}{i.e.~}	% id est = that is
\newcommand{\fe}{e.g.~}	% exempli gratia = for example
%not possible: \eg --> e.g.~
\newcommand{\vs}{vs.\ }	% versus
\newcommand{\wrt}{w.r.t.\ }	% with respect to


%%%%%%%%%%%%%%%%%%%%%%%%%%%%%%%%
%% TYPE SETTING
%%%%%%%%%%%%%%%%%%%%%%%%%%%%%%%%

%%%%%%%%%%%%%%%%%%%%%%%%%%%%%%%%
% German quotation marks:
\newcommand{\gqq}[1]{\glqq{}#1\grqq{}}		%double
\newcommand{\gq}[1]{\glq{}#1\grq{}}			%simple

%%%%%%%%%%%%%%%%%%%%%%%%%%%%%%%%
% Dash:
\newcommand{\gs}[1]{--\,#1\,--}

%%%%%%%%%%%%%%%%%%%%%%%%%%%%%%%%
%% Object- and Meta-language marking:
%\newcommand{\obj}[1]{\glqq{}#1\grqq{}}	             % German double quotes
%\newcommand{\obj}[1]{``#1''}					           % English double quotes
\newcommand{\obj}[1]{\emph{#1}}                          % Emphasising
\newcommand{\term}[1]{\emph{#1}}                       % for terminology
\newcommand{\termi}[1]{\index{#1}{\emph{#1}}}	  % for terminonoly + index

%%%%%%%%%%%%%%%%%%%%%%%%%%%%%%%%
% Size:
\newcommand{\size}[1]{{\footnotesize #1}}	% f.e. resize citations

%%%%%%%%%%%%%%%%%%%%%%%%%%%%%%%%
%% Short spacing 12 %, 5 cm, etc.
\newcommand{\sspace}[2]{#1\,#2}

%%%%%%%%%%%%%%%%%%%%%%%%%%%%%%%%
%% Small caps subscripts
\newcommand{\scdown}[1]{\textsubscript{\textsc{#1}}}

%%%%%%%%%%%%%%%%%%%%%%%%%%%%%%%%
% Writing text with colour:
% package needed: xcolor
% Command \alert{} in Beamer >> red
\newcommand{\blue}[1]{\textcolor{blue}{#1}}
\newcommand{\green}[1]{\textcolor{green}{#1}}
\newcommand{\red}[1]{\textcolor{red}{#1}}

%%%%%%%%%%%%%%%%%%%%%%%%%%%%%%%%
%% Marking text with colour: 
%%% package needed: color
\newcommand{\clrr}[1]{\colorbox{red}{#1}}
\newcommand{\clry}[1]{\colorbox{yellow}{#1}}


%%%%%%%%%%%%%%%%%%%%%%%%%%%%%%%%
%% LINGUISTIC TYPOGRAPHY
%%%%%%%%%%%%%%%%%%%%%%%%%%%%%%%%

%%%%%%%%%%%%%%%%%%%%%%%%%%%%%%%%
% Rightarrow & Leftarrow with and without space (--> <--)
\def\ra{\ensuremath\rightarrow}			%without space
\def\ras{\ensuremath\rightarrow\ }		%with space
\def\la{\ensuremath\leftarrow}
\def\las{\ensuremath\leftarrow\ }


%%%%%%%%%%%%%%%%%%%%%%%%%%%%%%%%
%% Semantic types (<e,t>), features, variables and graphemes in angled brackets 

%%% types and variables, in math mode: angled brackets + italics + no space
%\newcommand{\type}[1]{$<#1>$}

%%% OR more correctly: 
%%% Types and Variables: chevrons! + text in math mode (italics + no space)
\newcommand{\type}[1]{$\langle #1 \rangle$} %% In Math Mode, only single types
\newcommand{\typem}[1]{\langle #1 \rangle } %% Mathmode extra, complex types

%%% Features and Graphemes: chevrons! + normal font
\newcommand{\ab}[1]{$\langle$#1$\rangle$} %% no italics
\newcommand{\abe}[1]{$\langle$\emph{#1}$\rangle$} %% italics

%%% Meaning brackets ([[xyz]]) + expression in italics
%% package needed: MnSymbol
\newcommand{\sem}[1]{$\lsem$\emph{#1}$\rsem$} %% italics
\newcommand{\semm}[1]{\lsem \textrm{\emph{#1}} \rsem} %% in math mode, italics

%% Denotational set
%% use in math mode: \ds{\typem{e,t}}
\newcommand{\ds}[1]{\textrm{D}_{#1}}

%% Predicates in formulae
%% use in math mode: \pred{Tisch} 
%\newcommand{\pred}[1]{\textrm{#1}'}  %% simple predicate with prime
\newcommand{\pred}[1]{\texttt{#1}}  %% simple predicate in type writer
\newcommand{\predO}[1]{\textrm{\textsc{#1}}} %% Operator in small caps

%% Presuppositions
\newcommand{\prspp}{$\gg$} 

%% Implicature
\newcommand{\implc}{$+ \mkern-5mu >$} 

%%%%%%%%%%%%%%%%%%%%%%%%%%%%%%%%
%% Function symbol in Beamer Class!
%% italics and serif
\newcommand{\func}{\emph{\textrm{f}}}
\newcommand{\gunc}{\emph{\textrm{g}}}
\newcommand{\chiF}[1]{\chi _{\textrm{#1}}} 

%%%%%%%%%%%%%%%%%%%%%%%%%%%%%%%%
%% HPSG: Features and Values!
\newcommand{\wert}[1]{\emph{#1}}        %Values & Types
\newcommand{\feat}[1]{\textsc{#1}}       %Features
\newcommand{\inx}[1]{\fbox{\footnotesize #1}}       %HPSG indices

%%%%%%%%%%%%%%%%%%%%%%%%%%%%%%%%
%% Subscript & Superscript: no italics inside math mode
\newcommand{\down}[1]{\textsubscript{#1}}
\newcommand{\downm}[1]{_{\textrm{#1}}}

\newcommand{\up}[1]{\textsuperscript{#1}}
\newcommand{\upm}[1]{^{\textrm{#1}}}

%%%%%%%%%%%%%%%%%%%%%%%%%%%%%%%%%
%% Shorter Underline
\DeclareTextCommand{\_}{T1}{\leavevmode \kern.06em\vbox{\hrule width.4em}}

%%%%%%%%%%%%%%%%%%%%%%%%%%%%%%%%
%% (Syntactic) Trees
% package needed: forest
% with option: linguistics

%%% Setting for complex trees
%% "bottom word" is the old "sn edges", but the later is used as default (idk y)
\forestset{
	bottom word/.style={for tree={parent anchor=south, child anchor=north,align=center,base=bottom,where n children=0{tier=word,inner xsep=0pt,outer sep=0pt}{}}}, 
	%
	background tree/.style={for tree={text opacity=0.2,draw opacity=0.2,edge={draw opacity=0.2}}}
}

%% Use HideWd for translations when using roof and the translations are longer than the original text
%% https://tex.stackexchange.com/questions/167978/smaller-roofs-for-forest
\newcommand\HideWd[1]{%
	\makebox[0pt]{#1}%
}

%%%%%%%%%%%%%%%%%%%%%%%%%%%%%%%%
%% X-bar notation

%% Notation with primes (not emphasized): \xprime{X}
\newcommand{\xprime}[1]{#1$^{\prime}$}
\newcommand{\xxprime}[1]{#1$^{\prime\prime}$}
\newcommand{\xxxprime}[1]{#1$^{\prime\prime\prime}$}

%% Notation with primes (emphasized): \exbar{X}
\newcommand{\exprime}[1]{\emph{#1}$^{\prime}$}
\newcommand{\exxprime}[1]{\emph{#1}$^{\prime\prime}$}
\newcommand{\exxxprime}[1]{\emph{#1}$^{\prime\prime\prime}$}


%% Notation with zero and max (not emphasized): \xbar{X}
\newcommand{\xzero}[1]{#1$^{0}$}
\newcommand{\maxbar}[1]{#1$^{\textsc{max}}$}

%% Notation with zero and max (emphasized): \xbar{X}
\newcommand{\ezerobar}[1]{\emph{#1}$^{0}$}
\newcommand{\emaxbar}[1]{\emph{#1}$^{\textsc{max}}$}


%%%%%%%%%%%%%%%%%%%%%%%%%%%%%%%%
%% NOTES & BOX
%%%%%%%%%%%%%%%%%%%%%%%%%%%%%%%%

%%%%%%%%%%%%%%%%%%%%%%%%%%%%%%%%
%% Outputbox
\newcommand{\outputbox}[1]{\noindent\fbox{\parbox[t][][t]{0.98\linewidth}{#1}}\vspace{0.5em}}

%%%%%%%%%%%%%%%%%%%%%%%%%%%%%%%%
%% TO DO NOTES
% package needed: todonote
\newcommand{\todolit}[1]{\todo[color=green!40]{LIT: #1}}
\newcommand{\todored}[1]{\todo[color=red!60]{#1}}


%%%%%%%%%%%%%%%%%%%%%%%%%%%%%%%%
%% Chapter Notes: \begin{chnote}\end{chnote}
%%% package needed: 
%
\newenvironment{chnote}{
	\color{blue}
	%	\noindent \paragraph*{CHAPTER NOTES} 
	\begin{itemize*}
	}
	%	
	{
	\end{itemize*} 
}


%%%%%%%%%%%%%%%%%%%%%%%%%%%%%%%%
%% Index newcommand:
%%% package needed: imakeidx package
\newcommand{\is}[1]{\index{#1}{#1}}

%for terms of implementations (files, code, etc.)
\newcommand{\ist}[1]{\index{#1}{\texttt{#1}}}

%for terms in math mode
\newcommand{\ism}[1]{\index{$#1$}{$#1$}}   


%%%%%%%%%%%%%%%%%%%%%%%%%%%%%%%%
%% Theoretical Abbreviations & Commands
%% in English: 

%%%% Theories
\newcommand{\GB}{\index{Government \& Binding Theory (GB)}{GB}}	

\newcommand{\GPSG}{\index{Generalized Phrase Structure Grammar (GPSG)}{\ac{GPSG}}}	

%%%% HPSG Terminology & Commands
\newcommand{\attr}[1]{\index{attribute!\textsc{#1}}{\textsc{#1}}}
	

%%%% HPSG Principles:
\newcommand{\CaseP}{\index{principle!Case Principle (CaseP)}{CaseP}}


%%%% Further Terminology & Commands
\newcommand{\throle}{\index{theta role}{theta role}}
\newcommand{\throles}{\index{theta role}{theta roles}}


%%%%%%%%%%%%%%%%%%%%%%%%%%%%%%%%
%% Corpora
% 
\newcommand{\ESCOW}{\citetext{ESCOW \citeyear{SchaeferR&Co12a}}}