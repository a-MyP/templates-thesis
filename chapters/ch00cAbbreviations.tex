%%%%%%%%%%%%%%%%%%%%%%%%%%%%
%%%%%%%%%%%%%%%%%%%%%%%%%%%%
\chapter{Abbreviations}

%% Check the documentation of the package acronym
%% https://www.ctan.org/pkg/acronym
%%%%%%%%%%%%%%%%%%%%%%%%%%%%


All abbreviations used in this work -- except the ones for glosses in examples -- are listed below. For glossed examples, the norms and abbreviations supplied by the  \emph{Leipzig Glossing Rules} \citep[cf.][]{LeipzigGloss15a} were used.



\begin{multicols}{2}
\setlength{\columnseprule}{.5pt}
\begin{acronym}[n--sg--strong]
%
%A
	\acro{acc}[\textit{acc}]{accusative}
	\acro{ARG-ST}{argument-structure}
	\acro{AVM}{attribute-value-matrix}
%
%B
	\acro{BAG}{Bay Area Grammars}
%
%C
	\acro{CP}{complementiser phrase}
%
%D
	\acro{dat}[\textit{dat}]{dative}
	\acro{DTR}{daughter}	
%
%E
	\acro{EXP}{experiencer}
%
%F
%
%G
	\acro{gen}[\textit{gen}]{genitive}
	\acro{GPSG}{Generalized Phrase Structure Grammar}
%
%H
	\acro{HFC}{Head Feature Convention}	
%
%I
	\acro{IPA}{International Phonetic Alphabet}	
%
%L
	\acro{LEX-DTR}{lexical daughter}
	\acro{lxgen}[\textit{lx-gen}]{lexical-genitive}	
	\acro{lxnom}[\textit{lx-nom}]{lexical-nominative}
%
%M	
	\acro{MRS}{Minimal Recursion Semantics}
%
%N
	\acro{NP}{noun phrase}
%
%O
%
%P
%
%Q
	\acro{QP}{quantificational phrase}
%
%R
%
%S
	\acro{S}{sentence}
	\acro{SemP}{Semantics Principle}
%
%T
	\acro{TAG}{Tree Adjoining Grammar}
	\acro{TH}{theme}	
%
%U
%
%V
%
%W	
%
%X
%
%Y
%
%Z	
\end{acronym}
\end{multicols}
